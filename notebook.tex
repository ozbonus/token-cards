
% Default to the notebook output style

    


% Inherit from the specified cell style.




    
\documentclass[11pt]{article}

    
    
    \usepackage[T1]{fontenc}
    % Nicer default font (+ math font) than Computer Modern for most use cases
    \usepackage{mathpazo}

    % Basic figure setup, for now with no caption control since it's done
    % automatically by Pandoc (which extracts ![](path) syntax from Markdown).
    \usepackage{graphicx}
    % We will generate all images so they have a width \maxwidth. This means
    % that they will get their normal width if they fit onto the page, but
    % are scaled down if they would overflow the margins.
    \makeatletter
    \def\maxwidth{\ifdim\Gin@nat@width>\linewidth\linewidth
    \else\Gin@nat@width\fi}
    \makeatother
    \let\Oldincludegraphics\includegraphics
    % Set max figure width to be 80% of text width, for now hardcoded.
    \renewcommand{\includegraphics}[1]{\Oldincludegraphics[width=.8\maxwidth]{#1}}
    % Ensure that by default, figures have no caption (until we provide a
    % proper Figure object with a Caption API and a way to capture that
    % in the conversion process - todo).
    \usepackage{caption}
    \DeclareCaptionLabelFormat{nolabel}{}
    \captionsetup{labelformat=nolabel}

    \usepackage{adjustbox} % Used to constrain images to a maximum size 
    \usepackage{xcolor} % Allow colors to be defined
    \usepackage{enumerate} % Needed for markdown enumerations to work
    \usepackage{geometry} % Used to adjust the document margins
    \usepackage{amsmath} % Equations
    \usepackage{amssymb} % Equations
    \usepackage{textcomp} % defines textquotesingle
    % Hack from http://tex.stackexchange.com/a/47451/13684:
    \AtBeginDocument{%
        \def\PYZsq{\textquotesingle}% Upright quotes in Pygmentized code
    }
    \usepackage{upquote} % Upright quotes for verbatim code
    \usepackage{eurosym} % defines \euro
    \usepackage[mathletters]{ucs} % Extended unicode (utf-8) support
    \usepackage[utf8x]{inputenc} % Allow utf-8 characters in the tex document
    \usepackage{fancyvrb} % verbatim replacement that allows latex
    \usepackage{grffile} % extends the file name processing of package graphics 
                         % to support a larger range 
    % The hyperref package gives us a pdf with properly built
    % internal navigation ('pdf bookmarks' for the table of contents,
    % internal cross-reference links, web links for URLs, etc.)
    \usepackage{hyperref}
    \usepackage{longtable} % longtable support required by pandoc >1.10
    \usepackage{booktabs}  % table support for pandoc > 1.12.2
    \usepackage[inline]{enumitem} % IRkernel/repr support (it uses the enumerate* environment)
    \usepackage[normalem]{ulem} % ulem is needed to support strikethroughs (\sout)
                                % normalem makes italics be italics, not underlines
    

    
    
    % Colors for the hyperref package
    \definecolor{urlcolor}{rgb}{0,.145,.698}
    \definecolor{linkcolor}{rgb}{.71,0.21,0.01}
    \definecolor{citecolor}{rgb}{.12,.54,.11}

    % ANSI colors
    \definecolor{ansi-black}{HTML}{3E424D}
    \definecolor{ansi-black-intense}{HTML}{282C36}
    \definecolor{ansi-red}{HTML}{E75C58}
    \definecolor{ansi-red-intense}{HTML}{B22B31}
    \definecolor{ansi-green}{HTML}{00A250}
    \definecolor{ansi-green-intense}{HTML}{007427}
    \definecolor{ansi-yellow}{HTML}{DDB62B}
    \definecolor{ansi-yellow-intense}{HTML}{B27D12}
    \definecolor{ansi-blue}{HTML}{208FFB}
    \definecolor{ansi-blue-intense}{HTML}{0065CA}
    \definecolor{ansi-magenta}{HTML}{D160C4}
    \definecolor{ansi-magenta-intense}{HTML}{A03196}
    \definecolor{ansi-cyan}{HTML}{60C6C8}
    \definecolor{ansi-cyan-intense}{HTML}{258F8F}
    \definecolor{ansi-white}{HTML}{C5C1B4}
    \definecolor{ansi-white-intense}{HTML}{A1A6B2}

    % commands and environments needed by pandoc snippets
    % extracted from the output of `pandoc -s`
    \providecommand{\tightlist}{%
      \setlength{\itemsep}{0pt}\setlength{\parskip}{0pt}}
    \DefineVerbatimEnvironment{Highlighting}{Verbatim}{commandchars=\\\{\}}
    % Add ',fontsize=\small' for more characters per line
    \newenvironment{Shaded}{}{}
    \newcommand{\KeywordTok}[1]{\textcolor[rgb]{0.00,0.44,0.13}{\textbf{{#1}}}}
    \newcommand{\DataTypeTok}[1]{\textcolor[rgb]{0.56,0.13,0.00}{{#1}}}
    \newcommand{\DecValTok}[1]{\textcolor[rgb]{0.25,0.63,0.44}{{#1}}}
    \newcommand{\BaseNTok}[1]{\textcolor[rgb]{0.25,0.63,0.44}{{#1}}}
    \newcommand{\FloatTok}[1]{\textcolor[rgb]{0.25,0.63,0.44}{{#1}}}
    \newcommand{\CharTok}[1]{\textcolor[rgb]{0.25,0.44,0.63}{{#1}}}
    \newcommand{\StringTok}[1]{\textcolor[rgb]{0.25,0.44,0.63}{{#1}}}
    \newcommand{\CommentTok}[1]{\textcolor[rgb]{0.38,0.63,0.69}{\textit{{#1}}}}
    \newcommand{\OtherTok}[1]{\textcolor[rgb]{0.00,0.44,0.13}{{#1}}}
    \newcommand{\AlertTok}[1]{\textcolor[rgb]{1.00,0.00,0.00}{\textbf{{#1}}}}
    \newcommand{\FunctionTok}[1]{\textcolor[rgb]{0.02,0.16,0.49}{{#1}}}
    \newcommand{\RegionMarkerTok}[1]{{#1}}
    \newcommand{\ErrorTok}[1]{\textcolor[rgb]{1.00,0.00,0.00}{\textbf{{#1}}}}
    \newcommand{\NormalTok}[1]{{#1}}
    
    % Additional commands for more recent versions of Pandoc
    \newcommand{\ConstantTok}[1]{\textcolor[rgb]{0.53,0.00,0.00}{{#1}}}
    \newcommand{\SpecialCharTok}[1]{\textcolor[rgb]{0.25,0.44,0.63}{{#1}}}
    \newcommand{\VerbatimStringTok}[1]{\textcolor[rgb]{0.25,0.44,0.63}{{#1}}}
    \newcommand{\SpecialStringTok}[1]{\textcolor[rgb]{0.73,0.40,0.53}{{#1}}}
    \newcommand{\ImportTok}[1]{{#1}}
    \newcommand{\DocumentationTok}[1]{\textcolor[rgb]{0.73,0.13,0.13}{\textit{{#1}}}}
    \newcommand{\AnnotationTok}[1]{\textcolor[rgb]{0.38,0.63,0.69}{\textbf{\textit{{#1}}}}}
    \newcommand{\CommentVarTok}[1]{\textcolor[rgb]{0.38,0.63,0.69}{\textbf{\textit{{#1}}}}}
    \newcommand{\VariableTok}[1]{\textcolor[rgb]{0.10,0.09,0.49}{{#1}}}
    \newcommand{\ControlFlowTok}[1]{\textcolor[rgb]{0.00,0.44,0.13}{\textbf{{#1}}}}
    \newcommand{\OperatorTok}[1]{\textcolor[rgb]{0.40,0.40,0.40}{{#1}}}
    \newcommand{\BuiltInTok}[1]{{#1}}
    \newcommand{\ExtensionTok}[1]{{#1}}
    \newcommand{\PreprocessorTok}[1]{\textcolor[rgb]{0.74,0.48,0.00}{{#1}}}
    \newcommand{\AttributeTok}[1]{\textcolor[rgb]{0.49,0.56,0.16}{{#1}}}
    \newcommand{\InformationTok}[1]{\textcolor[rgb]{0.38,0.63,0.69}{\textbf{\textit{{#1}}}}}
    \newcommand{\WarningTok}[1]{\textcolor[rgb]{0.38,0.63,0.69}{\textbf{\textit{{#1}}}}}
    
    
    % Define a nice break command that doesn't care if a line doesn't already
    % exist.
    \def\br{\hspace*{\fill} \\* }
    % Math Jax compatability definitions
    \def\gt{>}
    \def\lt{<}
    % Document parameters
    \title{finding-letters}
    
    
    

    % Pygments definitions
    
\makeatletter
\def\PY@reset{\let\PY@it=\relax \let\PY@bf=\relax%
    \let\PY@ul=\relax \let\PY@tc=\relax%
    \let\PY@bc=\relax \let\PY@ff=\relax}
\def\PY@tok#1{\csname PY@tok@#1\endcsname}
\def\PY@toks#1+{\ifx\relax#1\empty\else%
    \PY@tok{#1}\expandafter\PY@toks\fi}
\def\PY@do#1{\PY@bc{\PY@tc{\PY@ul{%
    \PY@it{\PY@bf{\PY@ff{#1}}}}}}}
\def\PY#1#2{\PY@reset\PY@toks#1+\relax+\PY@do{#2}}

\expandafter\def\csname PY@tok@w\endcsname{\def\PY@tc##1{\textcolor[rgb]{0.73,0.73,0.73}{##1}}}
\expandafter\def\csname PY@tok@c\endcsname{\let\PY@it=\textit\def\PY@tc##1{\textcolor[rgb]{0.25,0.50,0.50}{##1}}}
\expandafter\def\csname PY@tok@cp\endcsname{\def\PY@tc##1{\textcolor[rgb]{0.74,0.48,0.00}{##1}}}
\expandafter\def\csname PY@tok@k\endcsname{\let\PY@bf=\textbf\def\PY@tc##1{\textcolor[rgb]{0.00,0.50,0.00}{##1}}}
\expandafter\def\csname PY@tok@kp\endcsname{\def\PY@tc##1{\textcolor[rgb]{0.00,0.50,0.00}{##1}}}
\expandafter\def\csname PY@tok@kt\endcsname{\def\PY@tc##1{\textcolor[rgb]{0.69,0.00,0.25}{##1}}}
\expandafter\def\csname PY@tok@o\endcsname{\def\PY@tc##1{\textcolor[rgb]{0.40,0.40,0.40}{##1}}}
\expandafter\def\csname PY@tok@ow\endcsname{\let\PY@bf=\textbf\def\PY@tc##1{\textcolor[rgb]{0.67,0.13,1.00}{##1}}}
\expandafter\def\csname PY@tok@nb\endcsname{\def\PY@tc##1{\textcolor[rgb]{0.00,0.50,0.00}{##1}}}
\expandafter\def\csname PY@tok@nf\endcsname{\def\PY@tc##1{\textcolor[rgb]{0.00,0.00,1.00}{##1}}}
\expandafter\def\csname PY@tok@nc\endcsname{\let\PY@bf=\textbf\def\PY@tc##1{\textcolor[rgb]{0.00,0.00,1.00}{##1}}}
\expandafter\def\csname PY@tok@nn\endcsname{\let\PY@bf=\textbf\def\PY@tc##1{\textcolor[rgb]{0.00,0.00,1.00}{##1}}}
\expandafter\def\csname PY@tok@ne\endcsname{\let\PY@bf=\textbf\def\PY@tc##1{\textcolor[rgb]{0.82,0.25,0.23}{##1}}}
\expandafter\def\csname PY@tok@nv\endcsname{\def\PY@tc##1{\textcolor[rgb]{0.10,0.09,0.49}{##1}}}
\expandafter\def\csname PY@tok@no\endcsname{\def\PY@tc##1{\textcolor[rgb]{0.53,0.00,0.00}{##1}}}
\expandafter\def\csname PY@tok@nl\endcsname{\def\PY@tc##1{\textcolor[rgb]{0.63,0.63,0.00}{##1}}}
\expandafter\def\csname PY@tok@ni\endcsname{\let\PY@bf=\textbf\def\PY@tc##1{\textcolor[rgb]{0.60,0.60,0.60}{##1}}}
\expandafter\def\csname PY@tok@na\endcsname{\def\PY@tc##1{\textcolor[rgb]{0.49,0.56,0.16}{##1}}}
\expandafter\def\csname PY@tok@nt\endcsname{\let\PY@bf=\textbf\def\PY@tc##1{\textcolor[rgb]{0.00,0.50,0.00}{##1}}}
\expandafter\def\csname PY@tok@nd\endcsname{\def\PY@tc##1{\textcolor[rgb]{0.67,0.13,1.00}{##1}}}
\expandafter\def\csname PY@tok@s\endcsname{\def\PY@tc##1{\textcolor[rgb]{0.73,0.13,0.13}{##1}}}
\expandafter\def\csname PY@tok@sd\endcsname{\let\PY@it=\textit\def\PY@tc##1{\textcolor[rgb]{0.73,0.13,0.13}{##1}}}
\expandafter\def\csname PY@tok@si\endcsname{\let\PY@bf=\textbf\def\PY@tc##1{\textcolor[rgb]{0.73,0.40,0.53}{##1}}}
\expandafter\def\csname PY@tok@se\endcsname{\let\PY@bf=\textbf\def\PY@tc##1{\textcolor[rgb]{0.73,0.40,0.13}{##1}}}
\expandafter\def\csname PY@tok@sr\endcsname{\def\PY@tc##1{\textcolor[rgb]{0.73,0.40,0.53}{##1}}}
\expandafter\def\csname PY@tok@ss\endcsname{\def\PY@tc##1{\textcolor[rgb]{0.10,0.09,0.49}{##1}}}
\expandafter\def\csname PY@tok@sx\endcsname{\def\PY@tc##1{\textcolor[rgb]{0.00,0.50,0.00}{##1}}}
\expandafter\def\csname PY@tok@m\endcsname{\def\PY@tc##1{\textcolor[rgb]{0.40,0.40,0.40}{##1}}}
\expandafter\def\csname PY@tok@gh\endcsname{\let\PY@bf=\textbf\def\PY@tc##1{\textcolor[rgb]{0.00,0.00,0.50}{##1}}}
\expandafter\def\csname PY@tok@gu\endcsname{\let\PY@bf=\textbf\def\PY@tc##1{\textcolor[rgb]{0.50,0.00,0.50}{##1}}}
\expandafter\def\csname PY@tok@gd\endcsname{\def\PY@tc##1{\textcolor[rgb]{0.63,0.00,0.00}{##1}}}
\expandafter\def\csname PY@tok@gi\endcsname{\def\PY@tc##1{\textcolor[rgb]{0.00,0.63,0.00}{##1}}}
\expandafter\def\csname PY@tok@gr\endcsname{\def\PY@tc##1{\textcolor[rgb]{1.00,0.00,0.00}{##1}}}
\expandafter\def\csname PY@tok@ge\endcsname{\let\PY@it=\textit}
\expandafter\def\csname PY@tok@gs\endcsname{\let\PY@bf=\textbf}
\expandafter\def\csname PY@tok@gp\endcsname{\let\PY@bf=\textbf\def\PY@tc##1{\textcolor[rgb]{0.00,0.00,0.50}{##1}}}
\expandafter\def\csname PY@tok@go\endcsname{\def\PY@tc##1{\textcolor[rgb]{0.53,0.53,0.53}{##1}}}
\expandafter\def\csname PY@tok@gt\endcsname{\def\PY@tc##1{\textcolor[rgb]{0.00,0.27,0.87}{##1}}}
\expandafter\def\csname PY@tok@err\endcsname{\def\PY@bc##1{\setlength{\fboxsep}{0pt}\fcolorbox[rgb]{1.00,0.00,0.00}{1,1,1}{\strut ##1}}}
\expandafter\def\csname PY@tok@kc\endcsname{\let\PY@bf=\textbf\def\PY@tc##1{\textcolor[rgb]{0.00,0.50,0.00}{##1}}}
\expandafter\def\csname PY@tok@kd\endcsname{\let\PY@bf=\textbf\def\PY@tc##1{\textcolor[rgb]{0.00,0.50,0.00}{##1}}}
\expandafter\def\csname PY@tok@kn\endcsname{\let\PY@bf=\textbf\def\PY@tc##1{\textcolor[rgb]{0.00,0.50,0.00}{##1}}}
\expandafter\def\csname PY@tok@kr\endcsname{\let\PY@bf=\textbf\def\PY@tc##1{\textcolor[rgb]{0.00,0.50,0.00}{##1}}}
\expandafter\def\csname PY@tok@bp\endcsname{\def\PY@tc##1{\textcolor[rgb]{0.00,0.50,0.00}{##1}}}
\expandafter\def\csname PY@tok@fm\endcsname{\def\PY@tc##1{\textcolor[rgb]{0.00,0.00,1.00}{##1}}}
\expandafter\def\csname PY@tok@vc\endcsname{\def\PY@tc##1{\textcolor[rgb]{0.10,0.09,0.49}{##1}}}
\expandafter\def\csname PY@tok@vg\endcsname{\def\PY@tc##1{\textcolor[rgb]{0.10,0.09,0.49}{##1}}}
\expandafter\def\csname PY@tok@vi\endcsname{\def\PY@tc##1{\textcolor[rgb]{0.10,0.09,0.49}{##1}}}
\expandafter\def\csname PY@tok@vm\endcsname{\def\PY@tc##1{\textcolor[rgb]{0.10,0.09,0.49}{##1}}}
\expandafter\def\csname PY@tok@sa\endcsname{\def\PY@tc##1{\textcolor[rgb]{0.73,0.13,0.13}{##1}}}
\expandafter\def\csname PY@tok@sb\endcsname{\def\PY@tc##1{\textcolor[rgb]{0.73,0.13,0.13}{##1}}}
\expandafter\def\csname PY@tok@sc\endcsname{\def\PY@tc##1{\textcolor[rgb]{0.73,0.13,0.13}{##1}}}
\expandafter\def\csname PY@tok@dl\endcsname{\def\PY@tc##1{\textcolor[rgb]{0.73,0.13,0.13}{##1}}}
\expandafter\def\csname PY@tok@s2\endcsname{\def\PY@tc##1{\textcolor[rgb]{0.73,0.13,0.13}{##1}}}
\expandafter\def\csname PY@tok@sh\endcsname{\def\PY@tc##1{\textcolor[rgb]{0.73,0.13,0.13}{##1}}}
\expandafter\def\csname PY@tok@s1\endcsname{\def\PY@tc##1{\textcolor[rgb]{0.73,0.13,0.13}{##1}}}
\expandafter\def\csname PY@tok@mb\endcsname{\def\PY@tc##1{\textcolor[rgb]{0.40,0.40,0.40}{##1}}}
\expandafter\def\csname PY@tok@mf\endcsname{\def\PY@tc##1{\textcolor[rgb]{0.40,0.40,0.40}{##1}}}
\expandafter\def\csname PY@tok@mh\endcsname{\def\PY@tc##1{\textcolor[rgb]{0.40,0.40,0.40}{##1}}}
\expandafter\def\csname PY@tok@mi\endcsname{\def\PY@tc##1{\textcolor[rgb]{0.40,0.40,0.40}{##1}}}
\expandafter\def\csname PY@tok@il\endcsname{\def\PY@tc##1{\textcolor[rgb]{0.40,0.40,0.40}{##1}}}
\expandafter\def\csname PY@tok@mo\endcsname{\def\PY@tc##1{\textcolor[rgb]{0.40,0.40,0.40}{##1}}}
\expandafter\def\csname PY@tok@ch\endcsname{\let\PY@it=\textit\def\PY@tc##1{\textcolor[rgb]{0.25,0.50,0.50}{##1}}}
\expandafter\def\csname PY@tok@cm\endcsname{\let\PY@it=\textit\def\PY@tc##1{\textcolor[rgb]{0.25,0.50,0.50}{##1}}}
\expandafter\def\csname PY@tok@cpf\endcsname{\let\PY@it=\textit\def\PY@tc##1{\textcolor[rgb]{0.25,0.50,0.50}{##1}}}
\expandafter\def\csname PY@tok@c1\endcsname{\let\PY@it=\textit\def\PY@tc##1{\textcolor[rgb]{0.25,0.50,0.50}{##1}}}
\expandafter\def\csname PY@tok@cs\endcsname{\let\PY@it=\textit\def\PY@tc##1{\textcolor[rgb]{0.25,0.50,0.50}{##1}}}

\def\PYZbs{\char`\\}
\def\PYZus{\char`\_}
\def\PYZob{\char`\{}
\def\PYZcb{\char`\}}
\def\PYZca{\char`\^}
\def\PYZam{\char`\&}
\def\PYZlt{\char`\<}
\def\PYZgt{\char`\>}
\def\PYZsh{\char`\#}
\def\PYZpc{\char`\%}
\def\PYZdl{\char`\$}
\def\PYZhy{\char`\-}
\def\PYZsq{\char`\'}
\def\PYZdq{\char`\"}
\def\PYZti{\char`\~}
% for compatibility with earlier versions
\def\PYZat{@}
\def\PYZlb{[}
\def\PYZrb{]}
\makeatother


    % Exact colors from NB
    \definecolor{incolor}{rgb}{0.0, 0.0, 0.5}
    \definecolor{outcolor}{rgb}{0.545, 0.0, 0.0}



    
    % Prevent overflowing lines due to hard-to-break entities
    \sloppy 
    % Setup hyperref package
    \hypersetup{
      breaklinks=true,  % so long urls are correctly broken across lines
      colorlinks=true,
      urlcolor=urlcolor,
      linkcolor=linkcolor,
      citecolor=citecolor,
      }
    % Slightly bigger margins than the latex defaults
    
    \geometry{verbose,tmargin=1in,bmargin=1in,lmargin=1in,rmargin=1in}
    
    

    \begin{document}
    
    
    \maketitle
    
    

    
    \section{Token Economy Cards}\label{token-economy-cards}

Basic description.

\subsection{Contents}\label{contents}

\begin{enumerate}
\def\labelenumi{\arabic{enumi}.}
\tightlist
\item
  Abstract
\item
  Card Design
\item
  Corpus

  \begin{enumerate}
  \def\labelenumii{\arabic{enumii}.}
  \tightlist
  \item
    Literature Corpus
  \item
    Target Word List
  \end{enumerate}
\item
  Letters and Digraphs
\item
  The Distribution

  \begin{enumerate}
  \def\labelenumii{\arabic{enumii}.}
  \tightlist
  \item
    Importing Libraries
  \item
    Counting Letters
  \item
    Plotting the Distribution
  \end{enumerate}
\end{enumerate}

    \subsection{1. Card Design}\label{card-design}

One half of a face will contain a letter or digraph (as well as a
preview of the letter on the other side). The other half of the face
will have a space a space for picture icons that will vary by type,
color, and number.

    \subsection{2. Corpus}\label{corpus}

Broadly define and explain the significance of a corpus.

    \paragraph{2A Literature Corpora}\label{a-literature-corpora}

Getting a corpus of writing aimed at children is more difficult than you
might imagine. Most books that children are likely to read today are
protected by copyright and so their full texts are hard to get a hold
of. Such corpora do exist, but they are not available to the public. It
turns out that grabbing (admittedly very old) books from Project
Gutenberg is the best option for my purposes.

Shibamouli Lahiri has already collected a bunch of works from Project
Gutenberg and done the arduous work of removing the project's licensing
and other stuff, making them suitable for language processing. The
dataset can be found here
https://web.eecs.umich.edu/\textasciitilde{}lahiri/gutenberg\_dataset.html

I picked, completely at my own whims, a selection of children's and
adult literature.

\begin{itemize}
\tightlist
\item
  Children's Literaure Corpus

  \begin{itemize}
  \tightlist
  \item
    Lewis Carroll, \emph{Alice's Adventures in Wonderland}
  \item
    Robert Louis Stevenson, \emph{Treasure Island}
  \item
    Oscar Wilde, \emph{The Happy Prince and Other Tales}
  \item
    Lyman Frank Baum, \emph{The Wonderful Wizard of Oz}
  \item
    Rudyard Kipling, \emph{The Jungle Book}
  \item
    Lucy Maud Montgomery, \emph{Anne Of Green Gables}
  \end{itemize}
\item
  Adult Literature Corpus

  \begin{itemize}
  \tightlist
  \item
    Bertrand Russell, \emph{Our Knowledge of the External World as a
    Field for Scientific Method in Philosophy}
  \item
    Herman \emph{Melville, Moby Dick}
  \item
    Jane Austen, \emph{Pride and Prejudice}
  \item
    Jonathan Swift, \emph{A Modest Proposal}
  \item
    Mary Shelley, \emph{Frankenstein; or the Modern Prometheus}
  \end{itemize}
\end{itemize}

    \begin{Verbatim}[commandchars=\\\{\}]
{\color{incolor}In [{\color{incolor}1}]:} \PY{n}{childrens\PYZus{}literature} \PY{o}{=} \PY{p}{[}\PY{l+s+s1}{\PYZsq{}}\PY{l+s+s1}{Lewis Carroll\PYZus{}\PYZus{}\PYZus{}Alice}\PY{l+s+se}{\PYZbs{}\PYZsq{}}\PY{l+s+s1}{s Adventures in Wonderland.txt}\PY{l+s+s1}{\PYZsq{}}\PY{p}{,}
                                \PY{l+s+s1}{\PYZsq{}}\PY{l+s+s1}{Lucy Maud Montgomery\PYZus{}\PYZus{}\PYZus{}Anne Of Green Gables.txt}\PY{l+s+s1}{\PYZsq{}}\PY{p}{,}
                                \PY{l+s+s1}{\PYZsq{}}\PY{l+s+s1}{Lyman Frank Baum\PYZus{}\PYZus{}\PYZus{}The Wonderful Wizard of Oz.txt}\PY{l+s+s1}{\PYZsq{}}\PY{p}{,}
                                \PY{l+s+s1}{\PYZsq{}}\PY{l+s+s1}{Oscar Wilde\PYZus{}\PYZus{}\PYZus{}The Happy Prince and Other Tales.txt}\PY{l+s+s1}{\PYZsq{}}\PY{p}{,}
                                \PY{l+s+s1}{\PYZsq{}}\PY{l+s+s1}{Robert Louis Stevenson\PYZus{}\PYZus{}\PYZus{}Treasure Island.txt}\PY{l+s+s1}{\PYZsq{}}\PY{p}{,}
                                \PY{l+s+s1}{\PYZsq{}}\PY{l+s+s1}{Rudyard Kipling\PYZus{}\PYZus{}\PYZus{}The Jungle Book.txt}\PY{l+s+s1}{\PYZsq{}}\PY{p}{]}
        
        \PY{n}{adult\PYZus{}literature} \PY{o}{=} \PY{p}{[}\PY{l+s+s1}{\PYZsq{}}\PY{l+s+s1}{Bertrand Russell\PYZus{}\PYZus{}\PYZus{}Our Knowledge of the External World as a Field for Scientific Method in Philosophy.txt}\PY{l+s+s1}{\PYZsq{}}\PY{p}{,}
                            \PY{l+s+s1}{\PYZsq{}}\PY{l+s+s1}{Herman Melville\PYZus{}\PYZus{}\PYZus{}Moby Dick.txt}\PY{l+s+s1}{\PYZsq{}}\PY{p}{,}
                            \PY{l+s+s1}{\PYZsq{}}\PY{l+s+s1}{Jane Austen\PYZus{}\PYZus{}\PYZus{}Pride and Prejudice.txt}\PY{l+s+s1}{\PYZsq{}}\PY{p}{,}
                            \PY{l+s+s1}{\PYZsq{}}\PY{l+s+s1}{Jonathan Swift\PYZus{}\PYZus{}\PYZus{}A Modest Proposal.txt}\PY{l+s+s1}{\PYZsq{}}\PY{p}{,}
                            \PY{l+s+s1}{\PYZsq{}}\PY{l+s+s1}{Mary Shelley\PYZus{}\PYZus{}\PYZus{}Frankenstein.txt}\PY{l+s+s1}{\PYZsq{}}\PY{p}{]}
\end{Verbatim}


    \paragraph{2B Target Word List Corpus}\label{b-target-word-list-corpus}

A significant part of my day job involves preparing primary school
students, who are learning English as s second language, to participate
in one of the Cambridge English Exams. These exams, as well as the
students I teach, cover a wide range of abilities. However, I believe
that it is the youngest students who will benefit the most from the sort
of game-like token economy item I am designing, so I would like to make
a mini-corpus from their official vocabulary list for young learners,
mostly out of curiosity.

\emph{FYI, I made the word list by processing a PDF with regex. It
caused mult-word terms (like sports center) to be split over multiple
lines. It's still good enough for counting letters, but don't try using
it as a teaching aid.}

    \begin{Verbatim}[commandchars=\\\{\}]
{\color{incolor}In [{\color{incolor}2}]:} \PY{n}{yle} \PY{o}{=} \PY{p}{[}\PY{l+s+s1}{\PYZsq{}}\PY{l+s+s1}{yle\PYZhy{}words.csv}\PY{l+s+s1}{\PYZsq{}}\PY{p}{]}
\end{Verbatim}


    \subsection{Letters and Digraphs}\label{letters-and-digraphs}

I've spent a lot of time playing word games with little kids who are
learning basic English vocabulary, and one trend I have noticed is that
they seldom produce words with consonant digraphs, despite having been
taught these words and successfully doing any number of quizzes and
worksheets demanding them to be spelled out. Perhaps this has something
to do with the interaction in their minds between English and their
native language (Chinese), but I don't have the expertise to weigh in on
why this happens. I just know that I want to encourage them to make
words with these features and I hope including cards with digraphs on
them will help with that.

    \begin{Verbatim}[commandchars=\\\{\}]
{\color{incolor}In [{\color{incolor}3}]:} \PY{n}{digraphs} \PY{o}{=} \PY{l+s+s1}{\PYZsq{}}\PY{l+s+s1}{bl cl fl br cr fr gl pl sl dr pr tr sm sn st sw ng nk sh ch th wh qu}\PY{l+s+s1}{\PYZsq{}}\PY{o}{.}\PY{n}{split}\PY{p}{(}\PY{p}{)}
\end{Verbatim}


    \subsection{The Distributions}\label{the-distributions}

Let's get started with counting letters.

    \paragraph{Importing Libraries}\label{importing-libraries}

The \texttt{Counter} class within the \texttt{collections} module of
Python's standard library is incredibly useful. It's a kind
special-purpose dictionary that, like it says on the tin, helps you
count things. It should be your first resort whenever you need to tally
something.

    \begin{Verbatim}[commandchars=\\\{\}]
{\color{incolor}In [{\color{incolor}4}]:} \PY{k+kn}{import} \PY{n+nn}{numpy} \PY{k}{as} \PY{n+nn}{np}
        \PY{k+kn}{import} \PY{n+nn}{pandas} \PY{k}{as} \PY{n+nn}{pd}
        \PY{k+kn}{from} \PY{n+nn}{collections} \PY{k}{import} \PY{n}{Counter}
        \PY{k+kn}{from} \PY{n+nn}{nltk}\PY{n+nn}{.}\PY{n+nn}{corpus} \PY{k}{import} \PY{n}{stopwords}
\end{Verbatim}


    \paragraph{The Counting Function}\label{the-counting-function}

I won't be satisfied by just running the books through a
\texttt{Counter} and calling it a day. Why? Consider the letter
\emph{h}. A lot of words contnain this letter: \emph{hello},
\emph{happy}, \emph{hippopotamus}, \emph{heliocentric}, \emph{hour},
etc. But if you take a look at a text, you'll see that \emph{h} is very
often a member of on of those digraphs I'm interested in: \emph{this},
\emph{what}, \emph{Thursday}, \emph{who}, etc. If we count every
occurence of \emph{th} and \emph{wh} and then count every occurnece of
\emph{h}, then the singular letter will be over-represented in the final
distribution because it will have been counted twice.

This counting function first tallies the occurences of each digraph and
then removes them from the text before counting the individual letters
that remain. Honestly, this could have been done with a nasty-looking
bit of regex, thus avoiding the extra step of removing characters, but
the corpora are small enough that this isn't a big deal, and besides,
trying to solve a problem with regex often just leaves you with two
problems to solve.

    \begin{Verbatim}[commandchars=\\\{\}]
{\color{incolor}In [{\color{incolor}5}]:} \PY{k}{def} \PY{n+nf}{count\PYZus{}digraphs\PYZus{}and\PYZus{}letters}\PY{p}{(}\PY{n}{corpus}\PY{p}{,} \PY{n}{digraphs}\PY{p}{,} \PY{n}{remove\PYZus{}stopwords}\PY{o}{=}\PY{k+kc}{False}\PY{p}{)}\PY{p}{:}
        
            \PY{n}{counter} \PY{o}{=} \PY{n}{Counter}\PY{p}{(}\PY{p}{)}
            \PY{n}{eng\PYZus{}stopwords} \PY{o}{=} \PY{n}{stopwords}\PY{o}{.}\PY{n}{words}\PY{p}{(}\PY{l+s+s1}{\PYZsq{}}\PY{l+s+s1}{english}\PY{l+s+s1}{\PYZsq{}}\PY{p}{)}
        
            \PY{k}{for} \PY{n}{item} \PY{o+ow}{in} \PY{n}{childrens\PYZus{}literature}\PY{p}{:}
                \PY{k}{with} \PY{n+nb}{open}\PY{p}{(}\PY{n}{item}\PY{p}{,} \PY{l+s+s1}{\PYZsq{}}\PY{l+s+s1}{r}\PY{l+s+s1}{\PYZsq{}}\PY{p}{)} \PY{k}{as} \PY{n}{file}\PY{p}{:}
                    \PY{n}{text} \PY{o}{=} \PY{n}{file}\PY{o}{.}\PY{n}{read}\PY{p}{(}\PY{p}{)}\PY{o}{.}\PY{n}{lower}\PY{p}{(}\PY{p}{)}
                    \PY{k}{if} \PY{n}{remove\PYZus{}stopwords}\PY{p}{:}
                        \PY{n}{text} \PY{o}{=} \PY{l+s+s1}{\PYZsq{}}\PY{l+s+s1}{ }\PY{l+s+s1}{\PYZsq{}}\PY{o}{.}\PY{n}{join}\PY{p}{(}\PY{p}{[}\PY{n}{word} \PY{k}{for} \PY{n}{word} \PY{o+ow}{in} \PY{n}{text}\PY{o}{.}\PY{n}{split}\PY{p}{(}\PY{p}{)} \PY{k}{if} \PY{n}{word} \PY{o+ow}{not} \PY{o+ow}{in} \PY{n}{eng\PYZus{}stopwords}\PY{p}{]}\PY{p}{)}
                \PY{k}{for} \PY{n}{digraph} \PY{o+ow}{in} \PY{n}{digraphs}\PY{p}{:}
                    \PY{n}{occurences} \PY{o}{=} \PY{n}{text}\PY{o}{.}\PY{n}{count}\PY{p}{(}\PY{n}{digraph}\PY{p}{)}
                    \PY{n}{counter}\PY{p}{[}\PY{n}{digraph}\PY{p}{]} \PY{o}{+}\PY{o}{=} \PY{n}{occurences}
                    \PY{n}{text} \PY{o}{=} \PY{n}{text}\PY{o}{.}\PY{n}{replace}\PY{p}{(}\PY{n}{digraph}\PY{p}{,} \PY{l+s+s1}{\PYZsq{}}\PY{l+s+s1}{ }\PY{l+s+s1}{\PYZsq{}}\PY{p}{)}
                \PY{n}{counter}\PY{o}{.}\PY{n}{update}\PY{p}{(}\PY{n}{char} \PY{k}{for} \PY{n}{char} \PY{o+ow}{in} \PY{l+s+s1}{\PYZsq{}}\PY{l+s+s1}{ }\PY{l+s+s1}{\PYZsq{}}\PY{o}{.}\PY{n}{join}\PY{p}{(}\PY{n}{text}\PY{p}{)} \PY{k}{if} \PY{n}{char} \PY{o+ow}{in} \PY{l+s+s1}{\PYZsq{}}\PY{l+s+s1}{abcdefghijklmnopqrstuvwxyz}\PY{l+s+s1}{\PYZsq{}}\PY{p}{)}
            
            \PY{n}{df} \PY{o}{=} \PY{n}{pd}\PY{o}{.}\PY{n}{DataFrame}\PY{o}{.}\PY{n}{from\PYZus{}dict}\PY{p}{(}\PY{n}{counter}\PY{p}{,} \PY{n}{orient}\PY{o}{=}\PY{l+s+s1}{\PYZsq{}}\PY{l+s+s1}{index}\PY{l+s+s1}{\PYZsq{}}\PY{p}{)}
            \PY{n}{df}\PY{o}{.}\PY{n}{reset\PYZus{}index}\PY{p}{(}\PY{n}{inplace}\PY{o}{=}\PY{k+kc}{True}\PY{p}{)}
            \PY{n}{df}\PY{o}{.}\PY{n}{columns} \PY{o}{=} \PY{p}{[}\PY{l+s+s1}{\PYZsq{}}\PY{l+s+s1}{letter}\PY{l+s+s1}{\PYZsq{}}\PY{p}{,} \PY{l+s+s1}{\PYZsq{}}\PY{l+s+s1}{count}\PY{l+s+s1}{\PYZsq{}}\PY{p}{]}
            \PY{n}{df}\PY{p}{[}\PY{l+s+s1}{\PYZsq{}}\PY{l+s+s1}{proportion}\PY{l+s+s1}{\PYZsq{}}\PY{p}{]} \PY{o}{=} \PY{n}{df}\PY{p}{[}\PY{l+s+s1}{\PYZsq{}}\PY{l+s+s1}{count}\PY{l+s+s1}{\PYZsq{}}\PY{p}{]} \PY{o}{/} \PY{n}{df}\PY{p}{[}\PY{l+s+s1}{\PYZsq{}}\PY{l+s+s1}{count}\PY{l+s+s1}{\PYZsq{}}\PY{p}{]}\PY{o}{.}\PY{n}{sum}\PY{p}{(}\PY{p}{)}
            \PY{n}{df}\PY{o}{.}\PY{n}{sort\PYZus{}values}\PY{p}{(}\PY{n}{by}\PY{o}{=}\PY{l+s+s1}{\PYZsq{}}\PY{l+s+s1}{count}\PY{l+s+s1}{\PYZsq{}}\PY{p}{,} \PY{n}{ascending}\PY{o}{=}\PY{k+kc}{False}\PY{p}{,} \PY{n}{inplace}\PY{o}{=}\PY{k+kc}{True}\PY{p}{)}
            \PY{n}{df}\PY{o}{.}\PY{n}{reset\PYZus{}index}\PY{p}{(}\PY{n}{inplace}\PY{o}{=}\PY{k+kc}{True}\PY{p}{,} \PY{n}{drop}\PY{o}{=}\PY{k+kc}{True}\PY{p}{)}
            
            \PY{k}{return} \PY{n}{df}
\end{Verbatim}


    \paragraph{Tally Ho!}\label{tally-ho}

Now it's finally time to count the digraphs and letters.

    \begin{Verbatim}[commandchars=\\\{\}]
{\color{incolor}In [{\color{incolor}39}]:} \PY{n}{c\PYZus{}dist} \PY{o}{=} \PY{n}{count\PYZus{}digraphs\PYZus{}and\PYZus{}letters}\PY{p}{(}\PY{n}{childrens\PYZus{}literature}\PY{p}{,} \PY{n}{digraphs}\PY{p}{)}
         \PY{n}{c\PYZus{}dist\PYZus{}r} \PY{o}{=} \PY{n}{count\PYZus{}digraphs\PYZus{}and\PYZus{}letters}\PY{p}{(}\PY{n}{childrens\PYZus{}literature}\PY{p}{,} \PY{n}{digraphs}\PY{p}{,} \PY{n}{remove\PYZus{}stopwords}\PY{o}{=}\PY{k+kc}{True}\PY{p}{)}
         \PY{n}{a\PYZus{}dist} \PY{o}{=} \PY{n}{count\PYZus{}digraphs\PYZus{}and\PYZus{}letters}\PY{p}{(}\PY{n}{adult\PYZus{}literature}\PY{p}{,} \PY{n}{digraphs}\PY{p}{)}
         \PY{n}{a\PYZus{}dist\PYZus{}r} \PY{o}{=} \PY{n}{count\PYZus{}digraphs\PYZus{}and\PYZus{}letters}\PY{p}{(}\PY{n}{adult\PYZus{}literature}\PY{p}{,} \PY{n}{digraphs}\PY{p}{,} \PY{n}{remove\PYZus{}stopwords}\PY{o}{=}\PY{k+kc}{True}\PY{p}{)}
         \PY{n}{y\PYZus{}dist} \PY{o}{=} \PY{n}{count\PYZus{}digraphs\PYZus{}and\PYZus{}letters}\PY{p}{(}\PY{n}{yle}\PY{p}{,} \PY{n}{digraphs}\PY{p}{)}
\end{Verbatim}


    \paragraph{The Super Scrabble
Distribution}\label{the-super-scrabble-distribution}

I'm interested in seeing how the professionals have already approached
this puzzle. I will compare my corpora with the distribution of letters
in a set of Super Scrabble. I chose this version rather then the
original because it was designed to make it easier to make more plurals
and third-person pronouns, which is something I would also like my
younger students to be more comfortable with.

    \begin{Verbatim}[commandchars=\\\{\}]
{\color{incolor}In [{\color{incolor}7}]:} \PY{n}{scrabble\PYZus{}letters} \PY{o}{=} \PY{n+nb}{list}\PY{p}{(}\PY{l+s+s1}{\PYZsq{}}\PY{l+s+s1}{eaotinrsludgcmbphfwyvkjxqz}\PY{l+s+s1}{\PYZsq{}}\PY{p}{)}
        \PY{n}{scrabble\PYZus{}counts}  \PY{o}{=} \PY{p}{[}\PY{l+m+mi}{24}\PY{p}{,} \PY{l+m+mi}{16}\PY{p}{,} \PY{l+m+mi}{15}\PY{p}{,} \PY{l+m+mi}{15} \PY{p}{,}\PY{l+m+mi}{13}\PY{p}{,} \PY{l+m+mi}{13}\PY{p}{,} \PY{l+m+mi}{13}\PY{p}{,} \PY{l+m+mi}{10}\PY{p}{,} \PY{l+m+mi}{7}\PY{p}{,} \PY{l+m+mi}{7}\PY{p}{,} \PY{l+m+mi}{8}\PY{p}{,} \PY{l+m+mi}{5}\PY{p}{,} \PY{l+m+mi}{6}\PY{p}{,} \PY{l+m+mi}{6}\PY{p}{,} \PY{l+m+mi}{4}\PY{p}{,} \PY{l+m+mi}{4}\PY{p}{,} \PY{l+m+mi}{5}\PY{p}{,} \PY{l+m+mi}{4}\PY{p}{,} \PY{l+m+mi}{4}\PY{p}{,} \PY{l+m+mi}{4}\PY{p}{,} \PY{l+m+mi}{3}\PY{p}{,} \PY{l+m+mi}{2}\PY{p}{,} \PY{l+m+mi}{2}\PY{p}{,} \PY{l+m+mi}{2}\PY{p}{,} \PY{l+m+mi}{2}\PY{p}{,} \PY{l+m+mi}{2}\PY{p}{]}
        
        \PY{n}{scrabble} \PY{o}{=} \PY{n}{pd}\PY{o}{.}\PY{n}{DataFrame}\PY{p}{(}\PY{p}{\PYZob{}}\PY{l+s+s1}{\PYZsq{}}\PY{l+s+s1}{letter}\PY{l+s+s1}{\PYZsq{}}\PY{p}{:} \PY{n}{scrabble\PYZus{}letters}\PY{p}{,} \PY{l+s+s1}{\PYZsq{}}\PY{l+s+s1}{count}\PY{l+s+s1}{\PYZsq{}}\PY{p}{:} \PY{n}{scrabble\PYZus{}counts}\PY{p}{\PYZcb{}}\PY{p}{)}
        \PY{n}{scrabble}\PY{p}{[}\PY{l+s+s1}{\PYZsq{}}\PY{l+s+s1}{proportion}\PY{l+s+s1}{\PYZsq{}}\PY{p}{]} \PY{o}{=} \PY{n}{scrabble}\PY{p}{[}\PY{l+s+s1}{\PYZsq{}}\PY{l+s+s1}{count}\PY{l+s+s1}{\PYZsq{}}\PY{p}{]} \PY{o}{/} \PY{n}{scrabble}\PY{p}{[}\PY{l+s+s1}{\PYZsq{}}\PY{l+s+s1}{count}\PY{l+s+s1}{\PYZsq{}}\PY{p}{]}\PY{o}{.}\PY{n}{sum}\PY{p}{(}\PY{p}{)}
        \PY{n}{scrabble} \PY{o}{=} \PY{n}{scrabble}\PY{o}{.}\PY{n}{reindex}\PY{p}{(}\PY{n}{columns}\PY{o}{=}\PY{p}{[}\PY{l+s+s1}{\PYZsq{}}\PY{l+s+s1}{letter}\PY{l+s+s1}{\PYZsq{}}\PY{p}{,} \PY{l+s+s1}{\PYZsq{}}\PY{l+s+s1}{count}\PY{l+s+s1}{\PYZsq{}}\PY{p}{,} \PY{l+s+s1}{\PYZsq{}}\PY{l+s+s1}{proportion}\PY{l+s+s1}{\PYZsq{}}\PY{p}{]}\PY{p}{)}
        \PY{n}{scrabble}\PY{o}{.}\PY{n}{sort\PYZus{}values}\PY{p}{(}\PY{n}{by}\PY{o}{=}\PY{l+s+s1}{\PYZsq{}}\PY{l+s+s1}{count}\PY{l+s+s1}{\PYZsq{}}\PY{p}{,} \PY{n}{ascending}\PY{o}{=}\PY{k+kc}{False}\PY{p}{,} \PY{n}{inplace}\PY{o}{=}\PY{k+kc}{True}\PY{p}{)}
\end{Verbatim}


    Now that all of the distributions are finished, let's take a look at
everything side-by-side.

    \begin{Verbatim}[commandchars=\\\{\}]
{\color{incolor}In [{\color{incolor}38}]:} \PY{n}{concat\PYZus{}letters} \PY{o}{=} \PY{n}{pd}\PY{o}{.}\PY{n}{concat}\PY{p}{(}\PY{p}{[}\PY{n}{c\PYZus{}dist}\PY{p}{,} \PY{n}{a\PYZus{}dist}\PY{p}{,} \PY{n}{y\PYZus{}dist}\PY{p}{,} \PY{n}{scrabble}\PY{p}{]}\PY{p}{,} \PY{n}{axis}\PY{o}{=}\PY{l+m+mi}{1}\PY{p}{)}
         \PY{n}{concat\PYZus{}letters}\PY{o}{.}\PY{n}{columns} \PY{o}{=} \PY{p}{[}\PY{l+s+s1}{\PYZsq{}}\PY{l+s+s1}{c\PYZus{}letter}\PY{l+s+s1}{\PYZsq{}}\PY{p}{,} \PY{l+s+s1}{\PYZsq{}}\PY{l+s+s1}{c\PYZus{}count}\PY{l+s+s1}{\PYZsq{}}\PY{p}{,} \PY{l+s+s1}{\PYZsq{}}\PY{l+s+s1}{c\PYZus{}proportion}\PY{l+s+s1}{\PYZsq{}}\PY{p}{,}
                                   \PY{l+s+s1}{\PYZsq{}}\PY{l+s+s1}{a\PYZus{}letter}\PY{l+s+s1}{\PYZsq{}}\PY{p}{,} \PY{l+s+s1}{\PYZsq{}}\PY{l+s+s1}{a\PYZus{}count}\PY{l+s+s1}{\PYZsq{}}\PY{p}{,} \PY{l+s+s1}{\PYZsq{}}\PY{l+s+s1}{a\PYZus{}proportion}\PY{l+s+s1}{\PYZsq{}}\PY{p}{,}
                                   \PY{l+s+s1}{\PYZsq{}}\PY{l+s+s1}{y\PYZus{}letter}\PY{l+s+s1}{\PYZsq{}}\PY{p}{,} \PY{l+s+s1}{\PYZsq{}}\PY{l+s+s1}{y\PYZus{}count}\PY{l+s+s1}{\PYZsq{}}\PY{p}{,} \PY{l+s+s1}{\PYZsq{}}\PY{l+s+s1}{y\PYZus{}proportion}\PY{l+s+s1}{\PYZsq{}}\PY{p}{,}
                                   \PY{l+s+s1}{\PYZsq{}}\PY{l+s+s1}{s\PYZus{}letter}\PY{l+s+s1}{\PYZsq{}}\PY{p}{,} \PY{l+s+s1}{\PYZsq{}}\PY{l+s+s1}{s\PYZus{}count}\PY{l+s+s1}{\PYZsq{}}\PY{p}{,} \PY{l+s+s1}{\PYZsq{}}\PY{l+s+s1}{s\PYZus{}proportion}\PY{l+s+s1}{\PYZsq{}}\PY{p}{,}\PY{p}{]}
         \PY{n}{concat\PYZus{}letters}\PY{o}{.}\PY{n}{head}\PY{p}{(}\PY{p}{)}
\end{Verbatim}


\begin{Verbatim}[commandchars=\\\{\}]
{\color{outcolor}Out[{\color{outcolor}38}]:}   c\_letter  c\_count  c\_proportion a\_letter  a\_count  a\_proportion y\_letter  \textbackslash{}
         0        e   153112      0.132885        e   153112      0.132885        e   
         1        a   105571      0.091625        a   105571      0.091625        a   
         2        o    94214      0.081768        o    94214      0.081768        o   
         3        i    82454      0.071561        i    82454      0.071561        i   
         4        n    71813      0.062326        n    71813      0.062326        n   
         
            y\_count  y\_proportion s\_letter  s\_count  s\_proportion  
         0   153112      0.132885        e     24.0      0.122449  
         1   105571      0.091625        a     16.0      0.081633  
         2    94214      0.081768        o     15.0      0.076531  
         3    82454      0.071561        t     15.0      0.076531  
         4    71813      0.062326        i     13.0      0.066327  
\end{Verbatim}
            
    Very interesting, but also difficult to interpret. Certainly there are
some differences, but they're hard to appreciate in this format. In the
next section I'll add a splash of color to illuminate the differences.

As an aside, I'm surpised to see that the two literature corpora both
have some occurences of \emph{q} outside of \emph{qu}. In English, this
usually only happens with loan words from other languages, like
\emph{qi} from Mandarin Chinese (e.g. 氣) or \emph{qabala} from Hebrew
where it is a Latinized version of the letter \emph{qoph}, ק. I don't
know which books this is happening in, but the far-flung adventures in a
few them certainly raises the opportunity to include a few loan words of
this nature. In the end, however, I'll definitely not have separate
cards for \emph{q} and \emph{qu}, but rather have them share a single
one.

    \paragraph{Visualizations}\label{visualizations}

Matplotlib is the standard library for making plots with Python and its
abilites are extended with Seaborn.

    \begin{Verbatim}[commandchars=\\\{\}]
{\color{incolor}In [{\color{incolor}9}]:} \PY{k+kn}{import} \PY{n+nn}{matplotlib}\PY{n+nn}{.}\PY{n+nn}{pyplot} \PY{k}{as} \PY{n+nn}{plt}
        \PY{k+kn}{import} \PY{n+nn}{seaborn} \PY{k}{as} \PY{n+nn}{sns}
        \PY{n}{sns}\PY{o}{.}\PY{n}{set}\PY{p}{(}\PY{n}{font}\PY{o}{=}\PY{l+s+s1}{\PYZsq{}}\PY{l+s+s1}{monospace}\PY{l+s+s1}{\PYZsq{}}\PY{p}{,} \PY{n}{style}\PY{o}{=}\PY{l+s+s1}{\PYZsq{}}\PY{l+s+s1}{white}\PY{l+s+s1}{\PYZsq{}}\PY{p}{)}
        \PY{o}{\PYZpc{}}\PY{k}{matplotlib} inline
\end{Verbatim}


    To highlight the differences between the distributions I'll create a
sequential color palette with colors mapped to the letters in the
children's literature corpus, sorted by occurences. Then I'll use that
palette, with colors still mapped to specific letters, on the other
distributions. Any differences will pop out by virtue of being out of
sequence.

    \begin{Verbatim}[commandchars=\\\{\}]
{\color{incolor}In [{\color{incolor}10}]:} \PY{n}{palette} \PY{o}{=} \PY{n}{sns}\PY{o}{.}\PY{n}{color\PYZus{}palette}\PY{p}{(}\PY{n}{palette}\PY{o}{=}\PY{l+s+s1}{\PYZsq{}}\PY{l+s+s1}{RdGy}\PY{l+s+s1}{\PYZsq{}}\PY{p}{,} \PY{n}{n\PYZus{}colors}\PY{o}{=}\PY{n}{c\PYZus{}dist}\PY{o}{.}\PY{n}{shape}\PY{p}{[}\PY{l+m+mi}{0}\PY{p}{]}\PY{p}{)}
         \PY{n}{letter\PYZus{}palette} \PY{o}{=} \PY{n}{pd}\PY{o}{.}\PY{n}{Series}\PY{p}{(}\PY{n}{palette}\PY{p}{,} \PY{n}{index}\PY{o}{=}\PY{n}{c\PYZus{}dist}\PY{p}{[}\PY{l+s+s1}{\PYZsq{}}\PY{l+s+s1}{letter}\PY{l+s+s1}{\PYZsq{}}\PY{p}{]}\PY{p}{)}\PY{o}{.}\PY{n}{to\PYZus{}dict}\PY{p}{(}\PY{p}{)}
         \PY{n}{sns}\PY{o}{.}\PY{n}{palplot}\PY{p}{(}\PY{n}{palette}\PY{p}{)}
\end{Verbatim}


    \begin{center}
    \adjustimage{max size={0.9\linewidth}{0.9\paperheight}}{output_24_0.png}
    \end{center}
    { \hspace*{\fill} \\}
    
    Plotting the absolute counts won't be useful because of the great size
differences between the corpora and the Super Scrabble distribution, but
plotting the proportions of occurences of each letter relative to the
sum of all of them will do fine.

The following will produce a rather large plot, so you might want to
open it in another window.

    \begin{Verbatim}[commandchars=\\\{\}]
{\color{incolor}In [{\color{incolor}42}]:} \PY{n}{fig}\PY{p}{,} \PY{n}{ax} \PY{o}{=} \PY{n}{plt}\PY{o}{.}\PY{n}{subplots}\PY{p}{(}\PY{n}{nrows}\PY{o}{=}\PY{l+m+mi}{6}\PY{p}{,} \PY{n}{figsize}\PY{o}{=}\PY{p}{(}\PY{l+m+mi}{15}\PY{p}{,} \PY{l+m+mi}{20}\PY{p}{)}\PY{p}{,} \PY{n}{facecolor}\PY{o}{=}\PY{l+s+s1}{\PYZsq{}}\PY{l+s+s1}{w}\PY{l+s+s1}{\PYZsq{}}\PY{p}{)}
         
         \PY{k}{for} \PY{n}{index}\PY{p}{,} \PY{n}{df} \PY{o+ow}{in} \PY{n+nb}{enumerate}\PY{p}{(}\PY{p}{[}\PY{n}{c\PYZus{}dist}\PY{p}{,} \PY{n}{c\PYZus{}dist\PYZus{}r}\PY{p}{,} \PY{n}{a\PYZus{}dist}\PY{p}{,} \PY{n}{a\PYZus{}dist\PYZus{}r}\PY{p}{,} \PY{n}{y\PYZus{}dist}\PY{p}{,} \PY{n}{scrabble}\PY{p}{]}\PY{p}{)}\PY{p}{:}
             \PY{n}{sns}\PY{o}{.}\PY{n}{barplot}\PY{p}{(}\PY{n}{data}\PY{o}{=}\PY{n}{df}\PY{p}{,} \PY{n}{x}\PY{o}{=}\PY{l+s+s1}{\PYZsq{}}\PY{l+s+s1}{letter}\PY{l+s+s1}{\PYZsq{}}\PY{p}{,} \PY{n}{y}\PY{o}{=}\PY{n}{df}\PY{p}{[}\PY{l+s+s1}{\PYZsq{}}\PY{l+s+s1}{proportion}\PY{l+s+s1}{\PYZsq{}}\PY{p}{]}\PY{p}{,} \PY{n}{palette}\PY{o}{=}\PY{n}{letter\PYZus{}palette}\PY{p}{,} \PY{n}{ax}\PY{o}{=}\PY{n}{ax}\PY{p}{[}\PY{n}{index}\PY{p}{]}\PY{p}{,} \PY{n}{linewidth}\PY{o}{=}\PY{l+m+mi}{1}\PY{p}{,} \PY{n}{edgecolor}\PY{o}{=}\PY{l+s+s1}{\PYZsq{}}\PY{l+s+s1}{.2}\PY{l+s+s1}{\PYZsq{}}\PY{p}{)}
             \PY{n}{ax}\PY{p}{[}\PY{n}{index}\PY{p}{]}\PY{o}{.}\PY{n}{set}\PY{p}{(}\PY{n}{xlabel}\PY{o}{=}\PY{l+s+s1}{\PYZsq{}}\PY{l+s+s1}{\PYZsq{}}\PY{p}{,} \PY{n}{ylabel}\PY{o}{=}\PY{l+s+s1}{\PYZsq{}}\PY{l+s+s1}{\PYZsq{}}\PY{p}{)}
         
         \PY{n}{ax}\PY{p}{[}\PY{l+m+mi}{0}\PY{p}{]}\PY{o}{.}\PY{n}{set\PYZus{}title}\PY{p}{(}\PY{l+s+s1}{\PYZsq{}}\PY{l+s+s1}{Children}\PY{l+s+se}{\PYZbs{}\PYZsq{}}\PY{l+s+s1}{s Lit}\PY{l+s+s1}{\PYZsq{}}\PY{p}{)}
         \PY{n}{ax}\PY{p}{[}\PY{l+m+mi}{1}\PY{p}{]}\PY{o}{.}\PY{n}{set\PYZus{}title}\PY{p}{(}\PY{l+s+s1}{\PYZsq{}}\PY{l+s+s1}{Children}\PY{l+s+se}{\PYZbs{}\PYZsq{}}\PY{l+s+s1}{s Lit without Stop Words}\PY{l+s+s1}{\PYZsq{}}\PY{p}{)}
         \PY{n}{ax}\PY{p}{[}\PY{l+m+mi}{2}\PY{p}{]}\PY{o}{.}\PY{n}{set\PYZus{}title}\PY{p}{(}\PY{l+s+s1}{\PYZsq{}}\PY{l+s+s1}{Adult Lit}\PY{l+s+s1}{\PYZsq{}}\PY{p}{)}
         \PY{n}{ax}\PY{p}{[}\PY{l+m+mi}{3}\PY{p}{]}\PY{o}{.}\PY{n}{set\PYZus{}title}\PY{p}{(}\PY{l+s+s1}{\PYZsq{}}\PY{l+s+s1}{Adult Lit without Stop Words}\PY{l+s+s1}{\PYZsq{}}\PY{p}{)}
         \PY{n}{ax}\PY{p}{[}\PY{l+m+mi}{4}\PY{p}{]}\PY{o}{.}\PY{n}{set\PYZus{}title}\PY{p}{(}\PY{l+s+s1}{\PYZsq{}}\PY{l+s+s1}{Cambridge YLE}\PY{l+s+s1}{\PYZsq{}}\PY{p}{)}
         \PY{n}{ax}\PY{p}{[}\PY{l+m+mi}{5}\PY{p}{]}\PY{o}{.}\PY{n}{set\PYZus{}title}\PY{p}{(}\PY{l+s+s1}{\PYZsq{}}\PY{l+s+s1}{Super Scrabble}\PY{l+s+s1}{\PYZsq{}}\PY{p}{)}
         
         \PY{n}{fig}\PY{o}{.}\PY{n}{tight\PYZus{}layout}\PY{p}{(}\PY{p}{)}
         \PY{n}{sns}\PY{o}{.}\PY{n}{despine}\PY{p}{(}\PY{p}{)}
         \PY{n}{plt}\PY{o}{.}\PY{n}{show}\PY{p}{(}\PY{p}{)}
\end{Verbatim}


    \begin{center}
    \adjustimage{max size={0.9\linewidth}{0.9\paperheight}}{output_26_0.png}
    \end{center}
    { \hspace*{\fill} \\}
    
    It shouldn't surprise anyone that some letters/digraphs are a lot more
common than others, but it's nevertheless amazing to see just how great
the differences can be. The letter \emph{j} doesn't feel all that
uncommon, but for every single one in the children'e literature corpus
the letter \emph{e} appears 83 times! Compare that to the distribution
of Super Scrabble letters where for every \emph{j} there are only 12
\emph{e}s.

Unfortunately, we can't see the color of many of the bars clearly,
stopping the plot from showing the different orders of the distributions
very well, so I'll make another plot where each letter/digraph gets the
same representation along the \(x\) axis.

    \begin{Verbatim}[commandchars=\\\{\}]
{\color{incolor}In [{\color{incolor}55}]:} \PY{n}{fig}\PY{p}{,} \PY{n}{ax} \PY{o}{=} \PY{n}{plt}\PY{o}{.}\PY{n}{subplots}\PY{p}{(}\PY{n}{nrows}\PY{o}{=}\PY{l+m+mi}{1}\PY{p}{,} \PY{n}{ncols}\PY{o}{=}\PY{l+m+mi}{6}\PY{p}{,} \PY{n}{figsize}\PY{o}{=}\PY{p}{(}\PY{l+m+mi}{10}\PY{p}{,} \PY{l+m+mi}{15}\PY{p}{)}\PY{p}{,} \PY{n}{facecolor}\PY{o}{=}\PY{l+s+s1}{\PYZsq{}}\PY{l+s+s1}{w}\PY{l+s+s1}{\PYZsq{}}\PY{p}{)}
         
         \PY{k}{for} \PY{n}{index}\PY{p}{,} \PY{n}{df} \PY{o+ow}{in} \PY{n+nb}{enumerate}\PY{p}{(}\PY{p}{[}\PY{n}{c\PYZus{}dist}\PY{p}{,} \PY{n}{c\PYZus{}dist\PYZus{}r}\PY{p}{,} \PY{n}{a\PYZus{}dist}\PY{p}{,} \PY{n}{a\PYZus{}dist\PYZus{}r}\PY{p}{,} \PY{n}{y\PYZus{}dist}\PY{p}{,} \PY{n}{scrabble}\PY{p}{]}\PY{p}{)}\PY{p}{:}
             \PY{c+c1}{\PYZsh{} x = proportion\PYZca{}0 will set all values to 1.}
             \PY{n}{sns}\PY{o}{.}\PY{n}{barplot}\PY{p}{(}\PY{n}{data}\PY{o}{=}\PY{n}{df}\PY{p}{,} \PY{n}{y}\PY{o}{=}\PY{l+s+s1}{\PYZsq{}}\PY{l+s+s1}{letter}\PY{l+s+s1}{\PYZsq{}}\PY{p}{,} \PY{n}{x}\PY{o}{=}\PY{n}{df}\PY{p}{[}\PY{l+s+s1}{\PYZsq{}}\PY{l+s+s1}{proportion}\PY{l+s+s1}{\PYZsq{}}\PY{p}{]}\PY{o}{*}\PY{o}{*}\PY{l+m+mi}{0}\PY{p}{,} \PY{n}{palette}\PY{o}{=}\PY{n}{letter\PYZus{}palette}\PY{p}{,} \PY{n}{ax}\PY{o}{=}\PY{n}{ax}\PY{p}{[}\PY{n}{index}\PY{p}{]}\PY{p}{,} \PY{n}{linewidth}\PY{o}{=}\PY{l+m+mi}{1}\PY{p}{,} \PY{n}{edgecolor}\PY{o}{=}\PY{l+s+s1}{\PYZsq{}}\PY{l+s+s1}{.2}\PY{l+s+s1}{\PYZsq{}}\PY{p}{)}
             \PY{n}{ax}\PY{p}{[}\PY{n}{index}\PY{p}{]}\PY{o}{.}\PY{n}{set}\PY{p}{(}\PY{n}{xlabel}\PY{o}{=}\PY{l+s+s1}{\PYZsq{}}\PY{l+s+s1}{\PYZsq{}}\PY{p}{,} \PY{n}{ylabel}\PY{o}{=}\PY{l+s+s1}{\PYZsq{}}\PY{l+s+s1}{\PYZsq{}}\PY{p}{,} \PY{n}{xticklabels}\PY{o}{=}\PY{l+s+s1}{\PYZsq{}}\PY{l+s+s1}{\PYZsq{}}\PY{p}{,} \PY{n}{yticklabels}\PY{o}{=}\PY{n}{df}\PY{p}{[}\PY{l+s+s1}{\PYZsq{}}\PY{l+s+s1}{letter}\PY{l+s+s1}{\PYZsq{}}\PY{p}{]}\PY{p}{)}
         
         \PY{n}{ax}\PY{p}{[}\PY{l+m+mi}{0}\PY{p}{]}\PY{o}{.}\PY{n}{set\PYZus{}title}\PY{p}{(}\PY{l+s+s1}{\PYZsq{}}\PY{l+s+s1}{Children}\PY{l+s+se}{\PYZbs{}\PYZsq{}}\PY{l+s+s1}{s}\PY{l+s+se}{\PYZbs{}n}\PY{l+s+s1}{Literature}\PY{l+s+s1}{\PYZsq{}}\PY{p}{)}
         \PY{n}{ax}\PY{p}{[}\PY{l+m+mi}{1}\PY{p}{]}\PY{o}{.}\PY{n}{set\PYZus{}title}\PY{p}{(}\PY{l+s+s1}{\PYZsq{}}\PY{l+s+s1}{w/o}\PY{l+s+se}{\PYZbs{}n}\PY{l+s+s1}{Stop Words}\PY{l+s+s1}{\PYZsq{}}\PY{p}{)}
         \PY{n}{ax}\PY{p}{[}\PY{l+m+mi}{2}\PY{p}{]}\PY{o}{.}\PY{n}{set\PYZus{}title}\PY{p}{(}\PY{l+s+s1}{\PYZsq{}}\PY{l+s+s1}{Adult}\PY{l+s+se}{\PYZbs{}n}\PY{l+s+s1}{Literature}\PY{l+s+s1}{\PYZsq{}}\PY{p}{)}
         \PY{n}{ax}\PY{p}{[}\PY{l+m+mi}{3}\PY{p}{]}\PY{o}{.}\PY{n}{set\PYZus{}title}\PY{p}{(}\PY{l+s+s1}{\PYZsq{}}\PY{l+s+s1}{w/o}\PY{l+s+se}{\PYZbs{}n}\PY{l+s+s1}{Stop Words}\PY{l+s+s1}{\PYZsq{}}\PY{p}{)}
         \PY{n}{ax}\PY{p}{[}\PY{l+m+mi}{4}\PY{p}{]}\PY{o}{.}\PY{n}{set\PYZus{}title}\PY{p}{(}\PY{l+s+s1}{\PYZsq{}}\PY{l+s+s1}{Cambridge}\PY{l+s+se}{\PYZbs{}n}\PY{l+s+s1}{Young Learner}\PY{l+s+se}{\PYZbs{}\PYZsq{}}\PY{l+s+s1}{s}\PY{l+s+s1}{\PYZsq{}}\PY{p}{)}
         \PY{n}{ax}\PY{p}{[}\PY{l+m+mi}{5}\PY{p}{]}\PY{o}{.}\PY{n}{set\PYZus{}title}\PY{p}{(}\PY{l+s+s1}{\PYZsq{}}\PY{l+s+s1}{Super}\PY{l+s+se}{\PYZbs{}n}\PY{l+s+s1}{Scrabble}\PY{l+s+s1}{\PYZsq{}}\PY{p}{)}
         
         
         \PY{n}{fig}\PY{o}{.}\PY{n}{tight\PYZus{}layout}\PY{p}{(}\PY{p}{)}
         \PY{n}{sns}\PY{o}{.}\PY{n}{despine}\PY{p}{(}\PY{n}{left}\PY{o}{=}\PY{k+kc}{True}\PY{p}{,} \PY{n}{bottom}\PY{o}{=}\PY{k+kc}{True}\PY{p}{)}
         \PY{n}{plt}\PY{o}{.}\PY{n}{show}\PY{p}{(}\PY{p}{)}
\end{Verbatim}


    \begin{center}
    \adjustimage{max size={0.9\linewidth}{0.9\paperheight}}{output_28_0.png}
    \end{center}
    { \hspace*{\fill} \\}
    
    The Cambridge Young Learner's English vocabulary list is the clear
outlier here. That's no surprise since every word appears only once
(with a few exceptions, e.g. \emph{bus}, \emph{bus station}). \emph{th},
for example, is much higher in other distributions because of the
frequent appearance of words like \emph{the}, \emph{this}, and
\emph{that}.


    % Add a bibliography block to the postdoc
    
    
    
    \end{document}
